\documentclass[a4paper,12pt]{article} % The document class with options
\usepackage[T1]{fontenc}
% A comment in the preamble

\title{The drivers of the level of COVID-19 vaccination in Poland}
\date{May 11, 2024}

\author{
    Tymoteusz Barciński\\
    Ziemowit Głowaczewski\\
    Jakub Bazyluk\\
    Antonina Ślubowska, MD\\
    \\
    Warsaw University of Technology,\\
    Faculty of Mathematics and Information Science
}

\begin{document}
% This is a comment

\maketitle

\section{Background}	
The COVID-19 pandemic, officially declared by the WHO in March 2020, has deeply affected global society, impacting public health, economics, and social dynamics. Healthcare systems worldwide faced shortages of vital resources, leading to the adoption of restrictive measures like lockdowns and social distancing protocols. While these measures aimed to control the spread of infections, they also disrupted social interactions and economic activities, causing widespread financial losses and job insecurity.
By May 2023, collaborative efforts between scientists and public health campaigners led to the development and deployment of effective vaccines, signaling the official end of the pandemic. Despite this milestone, the lessons learned from addressing COVID-19 vaccine hesitancy remain relevant. Vaccine hesitancy, once a significant challenge, may resurface in future, underscoring the need to understand its determinants for preparedness and response efforts in the years to come.
Positioned in Central and Eastern Europe, Poland offers a unique context for studying vaccination behavior, often overlooked compared to more extensively studied regions like the US and UK. Poland's historical experiences, socio-cultural dynamics, and political landscape significantly shape public health policies and vaccination strategies. Furthermore, analyzing regional disparities in vaccination rates within Poland can provide insights into the impact of socio-economic factors, healthcare infrastructure, and access to healthcare services. Despite the availability of COVID-19 vaccines, Poland has shown lower vaccination rates compared to the EU average, underscoring the importance of understanding the underlying reasons for this hesitancy. \cite{covid_vaccine_tracker}
By thoroughly examining these factors, policymakers and healthcare professionals can develop targeted interventions to effectively address vaccine hesitancy and enhance public health outcomes.

\section{Literature review}
One approach to assess the various factors contributing to vaccine hesitancy is through the application of psychological concepts like the 5C framework. This model identifies five key factors—confidence, complacency, convenience, risk calculation, and collective responsibility—that influence individuals' decisions regarding vaccination. \cite{10.1371/journal.pone.0208601}, \cite{Machingaidze2021}


\section{Aim of the study}
Our study aims to investigate the factors influencing COVID-19 vaccination rates at the municipal level in Poland. By analyzing vaccination data alongside socio-demographic, economic, and healthcare-related characteristics of municipalities, we seek to adapt the 5C model to formulate hypotheses about the municipality-level drivers of vaccination uptake. 
\subsection{Hypotheses}
\subsubsection{Confidence}
This component refers to trust in the effectiveness and safety of vaccines, as well as in the healthcare system and authorities that promote vaccination. Factors that can affect confidence include misinformation, mistrust in healthcare providers or pharmaceutical companies, and concerns about vaccine side effects.
As a potential measure of general trust in authorities, we considered the percentage of participants in parliamentary elections. Additionally, we investigated the voting patterns of residents in each municipality.
\\
We hypothesized that a higher voter turnout correlates with a higher vaccination rate, and that political affiliation impacts vaccination acceptance rates. 
\subsubsection{Complacency}
Complacency refers to the perception of the risk posed by vaccine-preventable diseases. When individuals perceive these diseases as low-risk or non-threatening, they may become less inclined to get vaccinated.
\\
Due to the increased risk that elderly individuals face from developing severe forms of COVID-19 and related complications, they exhibit greater willingness to receive vaccinations. Moreover, vaccination programs often prioritize this demographic. 
\\
Consequently, our hypothesis suggests that municipalities with a higher proportion of seniors will exhibit higher vaccination rates, while those with a higher proportion of individuals under the age of 20 will likely demonstrate lower vaccination rates.

\subsubsection{Convenience}
Convenience refers to the ease of access to vaccination services. Barriers to vaccination, such as long wait times, inconvenient clinic hours, or lack of transportation to vaccination sites, can reduce vaccine uptake. 
\\
Based on this, we hypothesize that the number of vaccination sites situated within a 10-kilometer radius of the municipality's center, along with the number of cars per 1000 inhabitants, will positively correlate with vaccination rates. Conversely, we anticipate that longer distances from the center of the municipality to the nearest vaccination point will negatively impact vaccination rates.

\subsubsection{Risk calculation}
Calculation refers to the process individuals use to weigh the risks and benefits of vaccination. Factors influencing this calculation include perceived vaccine efficacy, perceived severity of vaccine-preventable diseases, and perceived risk of vaccine side effects. Individuals may be more likely to accept vaccination if they perceive the benefits of vaccination to outweigh the risks.
\\
Based on this understanding, we hypothesize that a higher percentage of residents with higher education levels will positively correlate with vaccination rates. This assumption stems from the notion that education fosters a better understanding of the benefits of vaccination, thereby increasing acceptance rates.

\subsubsection{Collective responsibility}
Collective responsibility refers to the sense of duty individuals feel towards protecting the health of their community through vaccination. Factors influencing collective responsibility include social norms surrounding vaccination, perceived social pressure to vaccinate, and the belief that vaccination is a civic duty. Strengthening collective responsibility can help foster a culture of vaccination acceptance.
\\
This concept intersects with the hypotheses outlined in section 3.1.1.

\subsubsection{Other factors}
In addition to examining the 5C components, our study aimed to explore how the diverse characteristics of Polish municipalities relate to vaccination rates. These characteristics include population density, urban or rural classification, historical partitions of Poland's terrain, and income per capita, reflecting the economic diversity across Poland's regions.
\\
\\
Our hypothesis suggests that all of these factors play a role in shaping vaccination rates. Particularly, we anticipate a positive correlation between income per capita and vaccination rates, as higher income levels often improve access to vaccination services, thereby enhancing convenience aspect mentioned in 3.1.3.
\subsubsection{Spatial Analysis}

\bibliographystyle{plain} % We choose the "plain" reference style
\bibliography{refs}

\end{document}